% -----------------------------------------------
% Template for ISMIR 2013
% (based on earlier ISMIR templates)
% -----------------------------------------------

\documentclass{article}
\usepackage[utf8]{inputenc}
\usepackage{ismir2013,amsmath,cite}
\usepackage{graphicx}

% Title.
% ------
\title{CSC 475 Final Project: Design Specification}

% Single address
% To use with only one author or several with the same address
% ---------------
%\oneauthor
% {Names should be omitted for double-blind reviewing}
% {Affiliations should be omitted for double-blind reviewing}

% Two addresses
% --------------
%\twoauthors
% {First author} {School \\ Department}
% {Second author} {Company \\ Address}

% Three addresses
% --------------
\threeauthors
{Lee Gauthier} {\tt lgauthie@uvic.ca}
{Robert Janzen} {\tt robertjanzenbc@gmail.com}
{Sondra Moyls} {\tt smoyls@uvic.ca}

% Four addresses
% --------------
%\fourauthors
% {First author} {Affiliation1 \\ {\tt author1@ismir.edu}}
% {Second author}{Affiliation2 \\ {\tt author2@ismir.edu}}
% {Third author} {Affiliation3 \\ {\tt author3@ismir.edu}}
% {Fourth author} {Affiliation4 \\ {\tt author4@ismir.edu}}

\begin{document}
%
\maketitle
%

\section{Design Outline}\label{sec:desoutline}
Our project aims to explore the effects of data preprocessing on the accuracy
of chord recognition using hidden markov models. In particular, to what extent
can the extraction of transient information and harmonic information be used to
improve chord recognition? Are one of both of these techniques more efficient
for certain genres of music than for others? Previous research using these
preprocessing techniques is discussed in \cite{McVicar:00}.

\section{Tools}\label{sec:tools}

For this project we will use the annotated Billboard 100 data set. We will
append genre information to the data for tests on individual genres. For data
mining processes, we will use sci-kit. Feature extraction will be done using
Marsyas and Python

\section{Proposed Timeline}\label{sec:timeline}

Friday Feb 28 - Read all relevant papers, retrieve existing code \newline
Friday Mar 7 - Start working on F0 extraction and transient removal, create
more solid timeline \newline Friday Mar 14 - Continue working on feature
extraction, learning models, start progress report \newline Friday Mar 21 -
Progress Report Due \newline Friday Mar 28 - Final results, writing paper
\newline Friday April 4 - Presentation \newline Thursday April 10 - Final
Report Due

\section{Roles of Each Member}

To be determined in the following week.


\begin{thebibliography}{citations}

\bibitem {McVicar:00}
M. McVicar, et al:
``Automatic Chord Estimation for Audio: A Review of the State of the Art,''
{\it IEEE/ACM Transactions on Audio, Speech, and Language Processing},
Vol.~22,No.~2, pp.~1-20, 2014.

\bibitem{Ueda:01}
Yushi Ueda, et al. :
``HMM-Based Approach For Automatic Chord Detection Using Refined Acoustic
Features,''
{\it IEEE International Conference on Acoustics, Speech, and Signal
Processing 2010},
pp.~5518--5521, 2010.

\bibitem{Varewyck:02}
Matthias Varewyck, et al. :
``A Novel Chroma Representation of Polyphonic Music based on Multiple Pitch
Tracking Techniques,''
{\it 16th ACM International Conference on Multimedia},
2008.

\bibitem{Lee:03}
Kyogu Lee:
``Automatic Chord Recognition from Audio Using Enhanced Pitch Class Profile,''
{\it Center for Computer Research in Music and Acoustics, Stanford},
2006.

\bibitem{Papadopoulus:04}
Hélène Papadopoulos and George Tzanetakis:
``Modeling Chord and Key Structure With Markov Logic,''
{\it 13th International Society for Music Information Retrieval Conference},
pp. ~127-132, 2012.

\bibitem{Richardson:05}
Matthew Richardson and Pedro Domingos:
``Markov Logic Networks,''
{\it Department of Computer Science and Engineering, University of Washington,
Seattle, WA},
pp. ~1-43, 2006.

\bibitem{SciKit:06}
F. Pedregosa, et al.
"Scikit-learn: Machine Learning in Python,"
{\it Journal of Machine Learning Research},
Vol.~12,pp.~2825-2830, 2011.

\bibitem{Burgoyne:07}
John Ashley Burgoyne, Jonathan Wild, and Ichiro Fujinaga
"An Expert Gount-Truth Set For Audio Chord Recognition and Music Analysis,"
{\it 12th International Society for Music Information Retrieval Conference},
pp.~633-638, 2011.

\bibitem{Burgoyne:08}
Nanzhu Jiang et al.
"Analyzing Chroma Feature Types for Automated Chord Recognition,"
{\it AES 42nd International Conference },
pp.~1-10, 2011.

\end{thebibliography}

%\bibliography{ismir2013template}

\end{document}

